%Two resources useful for abstract writing.
% Guidance of how to write an abstract/summary provided by Nature: https://cbs.umn.edu/sites/cbs.umn.edu/files/public/downloads/Annotated_Nature_abstract.pdf %https://writingcenter.gmu.edu/guides/writing-an-abstract
\chapter*{\center \Large  Abstract}
%%%%%%%%%%%%%%%%%%%%%%%%%%%%%%%%%%%%%%
% Replace all text with your text
%%%%%%%%%%%%%%%%%%%%%%%%%%%%%%%%%%%

This is an undergraduate project report template and instruction on how to write a report. It also has some useful examples to use \LaTeX. Do read this template carefully. The number of chapters and their titles may vary depending on the type of project and personal preference. Section titles in this template are illustrative should be updated accordingly. For example, sections named ``A section...'' and ``Example of ...'' should be updated. The number of sections in each chapter may also vary. This template may or may not suit your project. Discuss the structure of your report with your supervisor.

%%%
~\\[1cm]%REMOVE THIS
\noindent\textbf{Guidance on abstract writing:} An abstract is a summary of a report in a single paragraph up to a maximum of 250 words. An abstract should be self-contained, and it should not refer to sections, figures, tables, equations, or references. An abstract typically consists of sentences describing the following four parts: (1) introduction (background and purpose of the project), (2) methods, (3) results and analysis, and (4) conclusions. The distribution of these four parts of the abstract should reflect the relative proportion of these parts in the report itself. An abstract starts with a few sentences describing the project's general field, comprehensive background and context, the main purpose of the project; and the problem statement. A few sentences describe the methods, experiments, and implementation of the project. A few sentences describe the main results achieved and their significance. The final part of the abstract describes the conclusions and the implications of the results to the relevant field.


%%%%%%%%%%%%%%%%%%%%%%%%%%%%%%%%%%%%%%%%%%%%%%%%%%%%%%%%%%%%%%%%%%%%%%%%%s
~\\[1cm]
\noindent % Provide your key words
\textbf{Keywords:} a maximum of five keywords/keyphrase separated by commas

\vfill
\noindent
\textbf{Report's total word count:} we expect a maximum of 20,000 words (excluding reference and appendices) and about 50 - 60 pages. [A good project report can also be written in approximately 10,000 words.]

